\documentclass{article}
\usepackage[utf8]{inputenc}
\usepackage{amsmath}
\usepackage{listings}
\usepackage{physics}

\title{Assignment 1
\\Linear Algebra }
\author{Swati Mohanty (EE20RESCH11007) }
\date{September 2020}

\begin{document}

\maketitle


\section{Problem}
Find the area of a rectangle $ABCD$ with vertices $A$ = \begin{pmatrix} -1  \\ \frac{1}{2} \\ 4 \end{pmatrix}, $B$ = \begin{pmatrix} 1  \\ \frac{1}{2} \\ 4 \end{pmatrix}, $C$ = \begin{pmatrix} 1  \\ -\frac{1}{2} \\ 4 \end{pmatrix}, $D$ = \begin{pmatrix} -1  \\ -\frac{1}{2} \\ 4 \end{pmatrix}.
\section{Solution}
Method 1: The adjacent sides of the rectangle are BA and AD (i.e. length and breadth). Area of a rectangle = length * breadth = AD*BA.
\\AD = A-D 
\\=\begin{pmatrix} -1  \\ \frac{1}{2} \\ 4 \end{pmatrix} - \begin{pmatrix} -1  \\ -\frac{1}{2} \\ 4 \end{pmatrix} = 1
\\Similarly, BA =  B-A   = 2. Thus, area = 1*2 = 2 sq.units
\\
\\Method 2: Area of rectangle = cross product of vectors of adjacent sides
\\Side $\vec{AD} $ = $\vec{A}$ - $\vec{D}$ = \begin{pmatrix}0 \\ -1 \\0\end{pmatrix}  Side $\vec{BA} $ = $\vec{B}$ - $\vec{A}$ = \begin{pmatrix}2\\ 0 \\0\end{pmatrix}
\\Area = $\vec{AD} $  X  $\vec{BA} $  = \begin{pmatrix}0 \\ -1 \\0\end{pmatrix}  X \begin{pmatrix}2\\ 0 \\0\end{pmatrix} 
\\=\begin{pmatrix}0&-0&1 \\0&0&0 \\-1&0&0\end{pmatrix}\begin{pmatrix}2\\ 0 \\0\end{pmatrix} = 2
\\
\\Python code link 
\begin{lstlisting}
https://github.com/Swati-Mohanty/EE5600/blob/master/
Assignment1/Code/quad_area.py
\end{lstlisting}
\end{document}
