\documentclass{article}
\usepackage[utf8]{inputenc}
\usepackage{amsmath}
\usepackage{listings}
\usepackage{physics}
\newcommand{\myvec}[1]{\ensuremath{\begin{pmatrix}#1\end{pmatrix}}}
\let\vec\mathbf

\title{Assignment 1
\\Linear Algebra }
\author{Swati Mohanty (EE20RESCH11007) }
\date{September 2020}

\begin{document}

\maketitle


\section{Problem}
Find the area of a rectangle $ABCD$ with vertices $A$ = \myvec{-1 \\\frac{1}{2} \\4} , $B$ = \myvec{1 \\\frac{1}{2} \\4}, $C$ = \myvec{1 \\-\frac{1}{2} \\4}, $D$ = \myvec{-1 \\-\frac{1}{2} \\4}.
\section{Solution}
Method 1: The adjacent sides of the rectangle are BA and AD (i.e. length and breadth). Area of a rectangle = length * breadth = AD*BA.
\\AD = $\Vec{A-D}$
\\=\myvec{-1 \\\frac{1}{2} \\4} - \myvec{-1 \\-\frac{1}{2} \\4} = 1
\\Similarly, BA =  $\vec{B-A}$   = 2. Thus, area = 1*2 = 2 sq.units
\\
\\Method 2: Area of rectangle = cross product of vectors of adjacent sides
\\Side $\vec{AD}$ = $\vec{A}$ - $\vec{D}$ = \myvec{0 \\-1 \\0}  Side $\vec{BA}$  = $\vec{B}$ - $\vec{A}$ = \myvec{2 \\0 \\0}
\\Area = $\vec{AD} $  X  $\vec{BA} $  = \myvec{0 \\-1 \\0}  X \myvec{2 \\0 \\0}
\\=\myvec{0&-0&1 \\0&0&0 \\-1&0&0}X\myvec{2 \\0 \\0} = 2
\\
\\Python code link 
\begin{lstlisting}
https://github.com/Swati-Mohanty/EE5600/blob/master/
Assignment1/Code/quad_area.py
\end{lstlisting}
\end{document}