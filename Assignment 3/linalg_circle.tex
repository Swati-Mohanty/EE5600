\documentclass[journal,12pt,twocolumn]{IEEEtran}
\usepackage[utf8]{inputenc}
\usepackage{amsmath}
\usepackage{amssymb}
\usepackage{listings}
\usepackage{physics}
\newcommand{\myvec}[1]{\ensuremath{\begin{pmatrix}#1\end{pmatrix}}}
\let\vec\mathbf
\usepackage{graphicx}
\graphicspath{ {./images/} }

\title{Assignment 3
\\Linear Algebra }
\author{Swati Mohanty (EE20RESCH11007) }
\date{October 2020}

\begin{document}

\maketitle


\section{Problem}
Find the equation of a circle with centre \myvec{2\\2} and passes through the point \myvec{4\\5}.
\section{Solution}
The equation of circle is given as
\begin{align}
    (x-h)^2 + (y-k)^2 = r^2
\end{align}
The centre is at \myvec{2\\2}, so h=2 and k=2. The equation now becomes
\begin{align}
    (x-2)^2 + (y-2)^2 = r^2
\end{align}
Since the circle passes through the point \myvec{4\\5}, this point is solution to the equation of circle. Substituting the values we get the radius as below:
\begin{align}
    r^2 = (4-2)^2 + (5-2)^2 
    \\\implies r = \sqrt{13}
\end{align}
Substituting the value of r in equation (2) and simplifying it we get the equation of circle.
\begin{align}
    (x-2)^2 + (y-2)^2 = \sqrt{13}^2
    \\\implies x^2 + 4 - 4x + y^2 + 4 - 4y = 13
    \\\implies x^2 + y^2 - 4(x+y) = 5
\end{align}
The following python code generates the equation of circle
\\Link : https://github.com/Swati-Mohanty/EE5600/blob/master/Assignment%203/codes/circle_equation.py
\end{document}